\documentclass[12 pt]{article}
\usepackage[utf8]{inputenc}
\usepackage[english]{babel}
\usepackage{amssymb}
\usepackage{amsmath}
\usepackage{amsthm}
\usepackage{amsfonts}
\usepackage{fullpage}
\usepackage[mathscr]{euscript}
\usepackage{youngtab}
\usepackage{graphicx}
\usepackage{color}
\usepackage{multirow}
\usepackage{enumerate}
\usepackage{mathrsfs}
\newcommand{\W}{\mathcal{W}}
\newcommand{\C}{\mathbb{C}}
\newcommand{\R}{\mathbb{R}}
\newcommand{\Z}{\mathbb{Z}}
\newcommand{\N}{\mathcal{N}}
\newcommand{\M}{\mathcal{M}}
\newcommand{\OO}{\mathcal{O}}
\newcommand{\B}{\mathcal{B}}
\newcommand{\pf}{\tilde{\phi}_N}
\newcommand{\g}{\mathfrak{g}}
\newcommand{\su}{\mathfrak{su}}
\newcommand{\so}{\mathfrak{so}}
\newcommand{\usp}{\mathfrak{usp}}
\newcommand{\h}{\mathfrak{h}}
\newcommand{\BB}{\mathfrak{B}}
\newcommand{\mon}{\OO_{\vec{n}}(a)}
\newcommand{\I}{\mathcal{I}}
%\numberwithin{equation}{section}
%\newtheorem{theorem}{Theorem}[section]
%\newtheorem{lemma}[theorem]{Lemma}
%\newtheorem{proposition}[theorem]{Proposition}
%\newtheorem{corollary}[theorem]{Corollary}

%\newenvironment{proof}[1][Proof]{\begin{trivlist}
%\item[\hskip \labelsep {\bfseries #1}]}{\end{trivlist}}
\newenvironment{definition}[1][Definition]{\begin{trivlist}
\item[\hskip \labelsep {\bfseries #1}]}{\end{trivlist}}
\newenvironment{example}[1][Example]{\begin{trivlist}
\item[\hskip \labelsep {\bfseries #1}]}{\end{trivlist}}
\newenvironment{remark}[1][Remark]{\begin{trivlist}
\item[\hskip \labelsep {\bfseries #1}]}{\end{trivlist}}
\begin{document}

\title{Chapter 2 Solutions}
\author{}
\date{}
\maketitle
\section{Linear Algebra}
\subsection{Bases and linear independence}
\begin{enumerate}
\item We see that 
$$a_1\begin{pmatrix}
	1\\-1
\end{pmatrix}+a_2\begin{pmatrix}
	1\\2
\end{pmatrix}+a_3\begin{pmatrix}
	2\\1
\end{pmatrix}=\begin{pmatrix}
	0\\0
\end{pmatrix}$$
has solution $a_1=a_2=-a_3 \neq 0$. Therefore the vectors are linearly dependent.
\item Choosing the basis $|0\rangle =\begin{pmatrix}
	1\\0
\end{pmatrix}, \ |1\rangle =\begin{pmatrix}
	0\\1
\end{pmatrix},$ the matrix representation of $A$ with respect to this basis is 
$$A=\begin{pmatrix}
	0&1\\1&0
\end{pmatrix}.$$
Taking instead the basis $|0\rangle=\frac{1}{\sqrt{2}}\begin{pmatrix}
	1\\1
\end{pmatrix}, \ |1\rangle=\frac{1}{\sqrt{2}}\begin{pmatrix}
	1\\-1
\end{pmatrix}$, the matrix representation of $A$ with respect to this basis is
$$A=\begin{pmatrix}
	1&0\\0&-1
\end{pmatrix}.$$
\item Using equation (2.12) repeatedly, we have 
	\begin{align*}
	BA|v_i\rangle &= B(\sum_j A_{ji}|w_j\rangle) \\
	&=\sum_j A_{ji}(B|w_j\rangle) \\
	&=\sum_j A_{ji} \sum_k B_{kj}|x_k\rangle \\
	&=\sum_k (\sum_j B_{kj}A_{ji})|x_k\rangle 
	\end{align*}
Also by (2.12), we have 
	\begin{align*}
	BA|v_i\rangle &=\sum_k (BA)_{ki}|x_k\rangle\end{align*}
Comparing the two expressions, we find
$$(BA)_{ki}=\sum_j B_{kj}A_{ji},$$
which is the matrix product of the matrix representations for $B$ and $A$.
\item Introducing the basis $|v_i\rangle$ for $V$, equation (2.12) gives
\begin{align*}
	I_V|v_j\rangle &=\sum_i (I_V)_{ij}|v_i\rangle \\
	&=|v_j\rangle
\end{align*}
which implies $(I_V)_{ij}=\delta_{ij}$.
\end{enumerate}
\end{document}