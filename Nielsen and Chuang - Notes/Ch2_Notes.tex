\documentclass[12 pt]{article}
\usepackage[utf8]{inputenc}
\usepackage[english]{babel}
\usepackage{amssymb}
\usepackage{amsmath}
\usepackage{amsthm}
\usepackage{amsfonts}
\usepackage{fullpage}
\usepackage[mathscr]{euscript}
\usepackage{youngtab}
\usepackage{graphicx}
\usepackage{color}
\usepackage{multirow}
\usepackage{enumerate}
\usepackage{mathrsfs}
\newcommand{\W}{\mathcal{W}}
\newcommand{\C}{\mathbb{C}}
\newcommand{\R}{\mathbb{R}}
\newcommand{\Z}{\mathbb{Z}}
\newcommand{\N}{\mathcal{N}}
\newcommand{\M}{\mathcal{M}}
\newcommand{\OO}{\mathcal{O}}
\newcommand{\B}{\mathcal{B}}
\newcommand{\pf}{\tilde{\phi}_N}
\newcommand{\g}{\mathfrak{g}}
\newcommand{\su}{\mathfrak{su}}
\newcommand{\so}{\mathfrak{so}}
\newcommand{\usp}{\mathfrak{usp}}
\newcommand{\h}{\mathfrak{h}}
\newcommand{\BB}{\mathfrak{B}}
\newcommand{\mon}{\OO_{\vec{n}}(a)}
\newcommand{\I}{\mathcal{I}}
%\numberwithin{equation}{section}
%\newtheorem{theorem}{Theorem}[section]
%\newtheorem{lemma}[theorem]{Lemma}
%\newtheorem{proposition}[theorem]{Proposition}
%\newtheorem{corollary}[theorem]{Corollary}

%\newenvironment{proof}[1][Proof]{\begin{trivlist}
%\item[\hskip \labelsep {\bfseries #1}]}{\end{trivlist}}
\newenvironment{definition}[1][Definition]{\begin{trivlist}
\item[\hskip \labelsep {\bfseries #1}]}{\end{trivlist}}
\newenvironment{example}[1][Example]{\begin{trivlist}
\item[\hskip \labelsep {\bfseries #1}]}{\end{trivlist}}
\newenvironment{remark}[1][Remark]{\begin{trivlist}
\item[\hskip \labelsep {\bfseries #1}]}{\end{trivlist}}
\begin{document}

\title{Chapter 2}
\author{Introduction to Quantum Mechanics}
\date{}
\maketitle
\section{Linear Algebra}
We consider vectors $|\psi\rangle \in V=\C^n$.
\subsection{Bases and linear independence}
\begin{itemize}
	\item A \emph{spanning set} for $V$ is a set of vectors $\{|v_1\rangle,\dots,|v_n\rangle\}$ such that $|v\rangle=\sum_ia_i|v_i\rangle$ for any $|v\rangle \in V$. \\
	\\
	$V$ may have many different spanning sets. E.g., for $\C^2$
	$$|v_1\rangle \equiv \begin{pmatrix}
		1\\0
	\end{pmatrix}, \ \ |v_2\rangle \equiv \begin{pmatrix}
		0\\1
	\end{pmatrix}$$
	and 
	$$|v_1\rangle \equiv \frac{1}{\sqrt{2}}\begin{pmatrix}
		1\\1
	\end{pmatrix}, \ \ \frac{1}{\sqrt{2}}|v_2\rangle \equiv \begin{pmatrix}
		1\\-1
	\end{pmatrix}$$
	are both spanning sets.
	\item A set of non-zero vectors $\{|v_1\rangle,\dots,|v_n\rangle\}$ are \emph{linearly independent} if 
	$$\sum_i a_i|v_i\rangle =0 \implies a_i=0 \ \forall i$$
	\item A \emph{basis} for $V$ is a linearly independent spanning set. 
	\item A basis always exists and any two bases for $V$ have the same number of elements. The number of elements in any basis for $V$ is called the \emph{dimension} of $V$. We will primarily be interested in finite-dimensional vector spaces.
\end{itemize}
\subsection{Linear operators and matrices}
\begin{itemize}
	\item A \emph{linear operator} is defined by
	\begin{align*}
		A&:V \to W \\
		A(\sum_i a_i|v_i\rangle)&=\sum_ia_i A|v_i\rangle
	\end{align*}
	It is clear that once the action of $A$ on a basis is specified, the action of $A$ is completely determined on all vectors.
	\item If $A:V \to W$ and $B:W \to X$, then the \emph{composition} of $B$ with $A$ is defined by
	\begin{align*}
		BA&:V \to X \\
		(BA)|v\rangle &=B(A|v\rangle)
	\end{align*}
	\item We can view a linear operator as a matrix $A:V \to W$ by choosing a basis for both $V$ and $W$. This is called the \emph{matrix representation} of $V$. Let $|v_i\rangle$ be a basis for $V$ and $|w_i\rangle$ be a basis for $W$. Then the matrix elements $A_{ij}$ are given by
	$$A|v_j\rangle = \sum_i A_{ij}|w_i\rangle$$
\end{itemize}
\end{document}